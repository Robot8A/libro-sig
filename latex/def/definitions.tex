\usepackage[T1]{fontenc}
\usepackage{libertine}
\usepackage{latexsym}
\usepackage{amssymb}
\usepackage{amsmath}
\usepackage[pdftex]{graphicx}
\usepackage[spanish]{babel}
\usepackage[latin1]{inputenc}
\usepackage{makeidx}
\usepackage{tabularx}
\usepackage{array}
\usepackage{hyperref}
\usepackage{xspace}
\usepackage{color}
\usepackage{booktabs}
\usepackage{paralist}
\usepackage{fancyhdr}
\usepackage{geometry}
\usepackage{bibentry}
\usepackage{multicol}
\usepackage{float}
\usepackage[raggedright]{titlesec}
\usepackage[protrusion=true,expansion=true]{microtype}
\usepackage[all]{nowidow}
\usepackage{etoolbox}

\makeatletter
\patchcmd{\@verbatim}
  {\verbatim@font}
  {\verbatim@font\small}
  {}{}
\makeatother


\renewcommand*\ttdefault{txtt}

\makeindex
\widowpenalty=9999

\renewcommand{\baselinestretch}{1}
\renewcommand{\enumerate}{\compactenum}
\setdefaultleftmargin{1mm}{}{}{}{}{}
\renewcommand{\itemize}{\compactitem}
\setlength{\pltopsep}{.1em}
\setlength{\plpartopsep}{.1em}

\geometry{verbose,dvips,paperwidth=7.44in,paperheight=9.68in,tmargin=2.6cm,bmargin=2.2cm, left=.8in, right=.6in}
\newcommand{\mycolumnwidth}{\textwidth}

\renewcommand{\chaptermark}[1]{\markboth{#1}{}}

\fancyhf{}
\fancyhead[LE,RO]{\textsc{\thepage}}
\fancyhead[RE]{\textsc{Sistemas de Informaci�n Geogr�fica}}
\fancyhead[LO]{\textsc{\leftmark}}
\fancypagestyle{plain}{%
\fancyhead{} 
\renewcommand{\headrulewidth}{0pt}
}

\numberwithin{equation}{section}
\renewcommand{\theequation}{\thesection.\arabic{equation}}

\makeatletter
\def\thickhrulefill{\leavevmode \leaders \hrule height 1ex \hfill \kern \z@}
\def\@makechapterhead#1{%
  \vspace*{10\p@}%
  {\parindent \z@
    {\raggedleft \reset@font
      \scshape \@chapapp{} \thechapter\par\nobreak}%
    \par\nobreak
    \vspace*{20\p@}
    \interlinepenalty\@M
    \begin{center}
    {\Large \bfseries #1}%
    \end{center}
    \par\nobreak
     \hrulefill
     \par
  }}
\def\@makeschapterhead#1{%
  \vspace*{10\p@}%
  {\parindent \z@
    {\raggedleft \reset@font
      \scshape \vphantom{\@chapapp{} \thechapter}\par\nobreak}%
    \par\nobreak
    \vspace*{20\p@}
    \interlinepenalty\@M
    {\Large \bfseries #1}\\
    \hrulefill
    \par    
  }}

\renewcommand{\v}[1]{\ensuremath{\overline{\mathbf{#1}}}}
\newcommand{\chapterauthor}[1]{\begin{center}\sffamily{#1}\end{center}}
\newcommand{\degree}{\ensuremath{^\circ}}
\newenvironment{intro}{\small\itshape}{\normalsize\par\nobreak\noindent\rule{\textwidth}{0.5pt}}

\definecolor{urlcolor}{rgb}{0,0,0.5}

\newcommand{\footurl}[1]{\footnote{\url{#1}\xspace}}
\newcommand{\lt}{<}
\newcommand{\gt}{>}

%%Configuraci�n de hyperref, SE RECOMIENDA DEJAR SIEMPRE AL FINAL DEL PRE�MBULO
\hypersetup{
    pdftitle=Sistemas de Informaci�n Geogr�fica,
    pdfpagelayout=TwoColumnRight,
    pdfproducer=PDFLaTeX,
    naturalnames,
    plainpages,
    colorlinks,
    linkcolor=urlcolor,
    urlcolor=urlcolor,
    citecolor=urlcolor,
    }